%%========================================================================
%% LaTeX sjabloon voor stage/projectrapport of bachelorproef
%%  HoGent Bedrijf en Organisatie
%%========================================================================

%%========================================================================
%% Preamble
%%========================================================================

\documentclass[pdftex,a4paper,12pt,twoside]{report}

% XXX: Let op: dit sjabloon is gemaakt om dubbelzijdig af te drukken
% Voor enkelzijdig, verwijder ``twoside'' hierboven.

%%---------- Extra functionaliteit ---------------------------------------

\usepackage[utf8]{inputenc}  % Accenten gebruiken in tekst (vb. é ipv \'e)
\usepackage{amsfonts}        % AMS math packages: extra wiskundige
\usepackage{amsmath}         %   symbolen (o.a. getallen-
\usepackage{amssymb}         %   verzamelingen N, R, Z, Q, etc.)
\usepackage[dutch]{babel}    % Taalinstellingen: woordsplitsingen,
                             %  commando's voor speciale karakters
                             %  ("dutch" voor NL)
\usepackage{eurosym}         % Euro-symbool €
\usepackage{geometry}
\usepackage{graphicx}        % Invoegen van tekeningen
\usepackage[pdftex,bookmarks=true]{hyperref}
                             % PDF krijgt klikbare links & verwijzingen,
                             %  inhoudstafel
\usepackage{listings}        % Broncode mooi opmaken
\usepackage{multirow}        % Tekst over verschillende cellen in tabellen
\usepackage{rotating}        % Tabellen en figuren roteren
\usepackage{natbib}          % Betere bibliografiestijlen
\usepackage{fancyhdr}        % Pagina-opmaak met hoofd- en voettekst

\usepackage[T1]{fontenc}     % Ivm lettertypes
\usepackage{lmodern}
\usepackage{textcomp}
\usepackage{listings}

\usepackage{lipsum}          % Voor vultekst (lorem ipsum)


\usepackage{glossaries}
\makenoidxglossaries

\newglossaryentry{tracker}
{
  name=tracker,
  description={Een soort van cookie waarbij een logboek van de online activiteiten van een gebruiker worden gekoppeld aan zijn Internet Protocol(IP) adres. De tracker stuurt deze informatie door naar een externe database voor analyse. Samen met gegevens van miljoenen anderen worden die gegevens dan gebruikt voor bijvoorbeeld marketing analyse. Deze tracker blijft werken ook al heeft men de site waar de tracker gedownload werd  al lang verlaten.  },
	plural={trackers},
}
\newglossaryentry{cross-site request}
{
  name=cross-site request,
  description={Een cross-site request is een request naar een volledig andere webpagina, die de browser in opdracht van een webpagina uitvoerd. },
	plural={cross-site requests}
}
\newglossaryentry{css selectors}
{
  name=css selector,
  description={In css worden selectors gebruikt om te selecteren welke delen moeten gebruik maken van een bepaalde stijl.},
	plural={css selectors}
}


%%---------- Layout ------------------------------------------------------

% hoofdingen, enz.
\pagestyle{fancy}
% enkel hoofdstuktitel in hoofding, geen sectietitel (vermijd overlap)
\renewcommand{\sectionmark}[1]{}

% lijn, wordt gebruikt in titelpagina
\newcommand{\HRule}{\rule{\linewidth}{0.5mm}}

% Leeg blad
\newcommand{\emptypage}{
\newpage
\thispagestyle{empty}
\mbox{}
\newpage
}

% Gebruik een schreefloos lettertype ipv het "oubollig" uitziende
% Computer Modern
\renewcommand{\familydefault}{\sfdefault}

% Commando voor invoegen Java-broncodebestanden (dank aan Niels Corneille)
% Gebruik: \codefragment{source/MijnKlasse.java}{Uitleg bij de code}
\newcommand{\codefragment}[2]{ \lstset{%
  language=java,
  breaklines=true,
  float=th,
  caption={#2},
  basicstyle=\scriptsize,
  frame=single,
  extendedchars=\true
}
\lstinputlisting{#1}}

%%---------- Documenteigenschappen ---------------------------------------
%% Vul dit aan met je eigen info:

% Je eigen naam
\newcommand{\student}{Tim Van Roosbroeck}

% De naam van je lector, begeleider, promotor
\newcommand{\promotor}{Sebastiaan Labijn}

% De naam van je co-promotor
\newcommand{\copromotor}{Joeri Van Steen}

% Indien je bachelorproef in opdracht van een bedrijf of organisatie
% geschreven is, geef je hier de naam.
\newcommand{\instelling}{---}

% De titel van het rapport/bachelorproef
\newcommand{\titel}{Adblocker en de impact op gratis content beschikbaar op het internet.}

% Datum van indienen
\newcommand{\datum}{29 mei 2016}

% Faculteit
\newcommand{\faculteit}{Faculteit Bedrijf en Organisatie}

% Soort rapport
\newcommand{\rapporttype}{Scriptie voorgedragen tot het bekomen van de graad van\\Bachelor in de toegepaste informatica}

% Academiejaar
\newcommand{\academiejaar}{2014-2015}

% Examenperiode
%  - 1e semester = 1e examenperiode
%  - 2e semester = 2e examenperiode
%  - tweede zit = 3e examenperiode
\newcommand{\examenperiode}{Tweede examenperiode}

%%========================================================================
%% Inhoud document
%%========================================================================

\begin{document}
%%---------- Front matter ------------------------------------------------
%% Het voorblad - Hier moet je in principe niets wijzigen.

\begin{titlepage}
\newgeometry{top=2cm,bottom=1.5cm,left=1.5cm,right=1.5cm}
  \begin{center}

    \begingroup
    \rmfamily
    \includegraphics[width=2.5cm]{img/HG-beeldmerk-woordmerk}\\[.5cm]
    \faculteit\\[3cm]
    \titel
    \vfill
    \student\\[3.5cm]
    \rapporttype\\[2cm]
    Promotor:\\
    \promotor\\
    Co-promotor:\\
    \copromotor\\[2.5cm]
    Instelling: \instelling\\[.5cm]
    Academiejaar: \academiejaar\\[.5cm]
    \examenperiode
    \endgroup

  \end{center}
  \restoregeometry
\end{titlepage}


% Schutblad

\emptypage


\begin{titlepage}
  \newgeometry{top=5.35cm,bottom=1.5cm,left=1.5cm,right=1.5cm}
  \begin{center}

    \begingroup
    \rmfamily
    \faculteit\\[3cm]
    \titel
    \vfill
    \student\\[3.5cm]
    \rapporttype\\[2cm]
    Promotor:\\
    \promotor\\
    Co-promotor:\\
    \copromotor\\[2.5cm]
    Instelling: \instelling\\[.5cm]
    Academiejaar: \academiejaar\\[.5cm]
    \examenperiode
    \endgroup

  \end{center}
  \restoregeometry
\end{titlepage}


\begin{abstract}
% TODO: De "abstract" of samenvatting is een kernachtige (max 1 blz. voor een
% thesis) synthese van het document. In ons geval beschrijf je kort de
% probleemstelling en de context, de onderzoeksvragen, de aanpak en de
% resultaten.

\end{abstract}

\chapter*{Voorwoord}
\label{ch:voorwoord}

% TODO: Vergeet ook niet te bedanken wie je geholpen/gesteund/... heeft


\tableofcontents

\printnoidxglossaries
% Als je een lijst van afkortingen of termen wil toevoegen, dan hoort die
% hier thuis. Gebruik bijvoorbeeld de ``glossaries'' package.

%%---------- Kern --------------------------------------------------------

\chapter{Inleiding}
\label{ch:inleiding}

Veel content op het internet is gratis te lezen of te bekijken. Denk maar aan filmpjes op YouTube, maar ook volledige programma’s van televisiezender alsook nieuwsdiensten zoals De Redactie, Het Laatste Nieuws en Sporza. Vaak staat daar reclame tegenover. Diensten van sites zoals Google of Facebook worden gratis aangeboden, maar tonen heel wat publiciteit op hun webpagina’s. Reclame is alom tegenwoordig op het internet.

Het inlassen van reclame op de websites gebeurt op verschillende manieren:

\begin{itemize}
	\item Een videofragment dat moet bekeken worden.

\item Een pop-up met reclame.

\item Een reclame blok doorheen of aan de rand van een artikel.

 \item …
\end{itemize}

Een adblocker is - meestal - een gratis plug-in voor browsers die er voor zorgt dat het overgrote deel van de reclameboodschappen geblokkeerd worden. Momenteel is Adblock Plus de meest gebruikte adblocker en ook de meest gedownloade browser extensie. De plug-in is beschikbaar voor negen browsers waaronder de vijf grootste. Sinds kort zijn adblockers ook beschikbaar voor smartphones.

Het gebruik van een adblocker zorgt voor een daling in reclame inkomsten voor content providers. Facebook en Google halen zelfs 90\% van hun inkomsten uit reclame en verliezen miljarden door adblockers. Ook makers van filmpjes op YouTube of Twitch verliezen inkomsten door adblockers.

% De inleiding moet de lezer alle nodige informatie verschaffen om het onderwerp te begrijpen zonder nog externe werken te moeten raadplegen \citep{Pollefliet2011}. Dit is een doorlopende tekst die gebaseerd is op al wat je over het onderwerp gelezen hebt (literatuuronderzoek).

% Je verwijst bij elke bewering die je doet, vakterm die je introduceert, enz. naar je bronnen. In \LaTeX{} kan dat met het commando \texttt{$\backslash${cite\{\}}} of \texttt{$\backslash${citep\{\}}}. Als argument van het commando geef je de ``sleutel'' van een ``record'' in een bibliografische databank in het Bib\TeX{}-formaat (een tekstbestand). Als je expliciet naar de auteur verwijst in de zin, gebruik je \texttt{$\backslash${}cite\{\}}.
% Soms wil je de auteur niet expliciet vernoemen, dan gebruik je \texttt{$\backslash${}citep\{\}}. Hieronder een voorbeeld van elk.

% \cite{Knuth1998} schreef een van de standaardwerken over sorteer- en zoekalgoritmen. Experten zijn het erover eens dat cloud computing een interessante opportuniteit vormen, zowel voor gebruikers als voor dienstverleners op vlak van informatietechnologie~\citep{Creeger2009}.

\section{Probleemstelling en Onderzoeksvragen}
\label{sec:onderzoeksvragen}

Het doel van het onderzoek is enerzijds inzicht te krijgen in de werking van adblockers en anderzijds het bepalen van de impact en de gevolgen op korte en lange termijn op de bedrijven die hun inkomsten halen uit reclame.
\begin{itemize}
	\item Wat is de impact, op dit moment, van adblockers op reclame inkomsten? Wie lijdt er het meeste onder?
\item Wat zouden de gevolgen kunnen zijn van een adblocker op langere termijn voor diensten zoals YouTube, Google en Facebook? Welke zijn de gevolgen de gevolgen voor de “kleine” content providers die bijvoorbeeld op YouTube of Twitch content aanbieden?
\item Op welke manieren omzeilen content providers momenteel adblockers? Hoe zullen ze zich in de toekomst wapenen tegen adblockers
\end{itemize}
?
% TODO: Wees zo concreet mogelijk bij het formuleren van je
% onderzoeksvra(a)g(en). Een onderzoeksvraag is trouwens iets waar nog
% niemand op dit moment een antwoord heeft (voor zover je kan nagaan).


\chapter{Methodologie}
\label{ch:methodologie}

% TODO: Hoe ben je te werk gegaan? Verdeel je onderzoek in grote fasen, en
% licht in elke fase toe welke stappen je gevolgd hebt. Verantwoord waarom je
% op deze manier te werk gegaan bent. Je moet kunnen aantonen dat je de best
% mogelijke manier toegepast hebt om een antwoord te vinden op de
% onderzoeksvraag.


\chapter{Wat is een adblocker}
\label{ch:Wat is een adblocker}
De term ad blocker is de overkoepelende term voor alle soorten software die advertenties verwijderen op een webpagina. Naast het blokkeren van reclame hebben de meest gebruikte ad blockers zoals Adblock Plus en AdBlock nog andere functionaliteiten. Voorbeelden hiervan zijn het uitschakelen van tracking, en het blokkeren van Malware domeinen. Het blokkeren van advertenties zorgt niet alleen voor dat de gebruiker een aangenamere ervaring heeft. Een ad blocker zorgt er ook voor dat webpagina's sneller geladen worden end dat er minder bandbreedte wordt verbruikt. Doordat reclame wordt geblokkeerd zullen reclame bedrijven ook minder informatie over ad blocker gebruikers ter beschikking hebben. Met als gevold dat de privacy beter wordt beschermd. Door het uitschakelen van \glspl{tracker} gaan ad blockers nog een stap verder in het beschermen van de privacy van hun gebruikers.

\section{Adblock Plus}
\label{sec:Adblock Plus}
Adblock Plus \footnote{\url{https://adblockplus.org/}} is gratis extensie voor Internet Explorer, Mozilla Firefox, Google Chrome, Opera en Safari. Eyeo GmbH is het bedrijf achter Adblock Plus. Over alle browsers heen is dit de meest gebruikte ad blocker met in totaal meer dan 50 miljoen gebruikers \footnote{\url{https://chrome.google.com/webstore/search/ad\%20block?utm_source=chrome-ntp-icon&_category=extensions}} \footnote{\url{https://addons.mozilla.org/nl/firefox/extensions/?sort=users}}. Anders dan andere ad blockers haalt Adblock Plus zijn inkomsten niet uit donaties. Met hun "`Acceptable ads program"' laten ze bedrijven betalen om hun advertenties op een witte lijst te plaatsen. 
\section{AdBlock}
\label{sec:AdBlock}
Adblock \footnote{\url{https://getadblock.com/}} is een gratis en opensource extensie voor Google Chrome en Safari. Het team achter AdBlock haalt zijn inkomsten volledig uit donaties. Met in totaal 40 miljoen gebruikers is dit de op een na grootste ad blocker.

\section{µBlock Origin}
\label{sec:uBlock Origin}%nog aanpassen
µBlock Origin \footnote{\url{https://github.com/gorhill/uBlock/}} is ook een gratis en opensource extensie voor Google Chrome en Mozilla Firefox. µBlock werkt volledig zonder donaties. Benchmarks door µBlock zelf en \citep{PerformanceAB} tonen aan dat µBlock sneller werkt met een lagere belasting van de CPU. µBlock kan er ook voor zorgen dat scripts enkel uitgevoerd kunnen worden door vertrouwde sites. En het is ook mogelijk om \glspl{cross-site request} te blokkeren.


%nog iets zeggen van native ad-blocking en adblocking op smartphones, door opera

\chapter{Hoe werkt een adblocker}
\label{ch:Hoe werkt een adblocker}
Ad blockers zoals AdBlock en Adblock Plus hebben op zichzelf geen functionaliteit, ze moeten verteld worden wat geblokkeerd moet worden. Dit wordt mogelijk gemaakt door externe filters toe te voegen. Filters zijn in wezen een uitgebreide set van regels die een ad blocker vertellen welke elementen geblokkeerd moeten worden. De meest gebruikte filter is die van EasyList\footnote{\url{https://easylist.adblockplus.org/nl/}}. Alle eerder vernoemde ad blockers gebruiken als basis een filter van EasyList. Deze filters zijn deels regio gebonden, een Nederlandstalige filter zal een grotere focus hebben op Nederlandse (en ook Vlaamse) domeinen. Filters zijn regels tekst die bepalen welke adressen niet mogen worden geladen of welke HTML DOM Elementen niet mogen worden getoond.

\section{Request blocking}
\label{sec:Request blocking}
De meeste content wordt geblokkeerd door request blocking waarbij ad blockers met behulp van de filters bepalen welke HTTP of HTTPS requests onderschept moeten worden. De Filters zijn opgebouwd vanuit eenvoudige regels.
\subsection{Basis regels}
\label{sec Basis regels}
De meest eenvoudige filter die kan gedefinieerd kan worden is bijvoorbeeld voor deze reclameblok \texttt{http://voorbeeld.com/ads/banner123.jpg}. Je kan deze regel al makkelijk beter maken door alle banners te blokkeren \texttt{voorbeeld.com/ads/banner*.jpg} of nog beter\texttt{ http://voorbeeld.com/ads/*}. Standaard zullen ad blockers voor en na elke regel een wildcard \texttt{*} zetten. De filters \texttt{advertentie}, \texttt{ad} en \texttt{*ad*} zijn dan alle drie gelijk. Het is belangrijk hier rekening mee te houden wanneer men bijvoorbeeld alle flash bestanden wil blokkeren zou men de filter \texttt{swf} kunnen toepassen. Maar deze filter zal er ook voor zorgen dat \texttt{http://voorbeeld.com/swf/index.html} zal geblokkeerd worden. De oplossing voor dit probleem is het pipe sympool |, \texttt{swf|} zal enkel adressen blokkeren die eindigen met swf. Wanneer men zowel \texttt{\textbf{http://}voorbeeld.com/banner.jpg}, \texttt{\textbf{https://}voorbeeld.com/banner.jpg} als \texttt{\textbf{https://}voorbeeld.com/banner.jpg} geblokkeerd moeten worden kan men gebruik maken van het dubbele pipe symbool: ||. \texttt{||voorbeeld.com/banner.jpg} zal de drie voorgaande adressen blokkeren.
Naast de basis regels zijn er nog meer geavanceerde regels\footnote{\url{https://adblockplus.org/en/filters}} of kan er gewerkt worden met reguliere expressies. Het gebruik van reguliere expressies wordt sterk afgeraden omdat dit de performantie naar omlaag haalt.
\subsection{Uitzonderingen}
\label{sec Uitzonderingen}
Wanneer filters die in de meeste gevallen goed werken toch adressen blokkeren die niet zouden mogen geblokkeerd worden dan kunnen uitzonderingen gebruikt worden. Als de filter \texttt{adv} gebruikt wordt en \texttt{http://voorbeeld.com/advice.html} mag niet geblokkeerd worden dan kan men als uitzondering \texttt{@@advice} gebruiken. Uitzonderingsregels verschillen voor de rest niet van de filter regels en kunnen dus op dezelfde manier samengesteld worden.
\section{ Element hidding}
\label{sec:Element hidding}
Wanneer reclame niet kan geblokkeerd worden door request blocking, bijvoorbeeld omdat die reclame meekomt als tekst samen met de werkelijk opgevraagde webpagina, dan kan men gebruik maken van element hidding. Element hidding kan gebruikt worden als een deel van de broncode van de webpagina er bijvoorbeeld zo uit ziet:
\lstset{language=Html,tabsize=2}  
\begin{lstlisting}
	<div class="textad">
		Goedkoop bier, enkel hier en nu!
	</div>
	<div id="sponsorad">
		Goedkope pizza, klik hier!
	</div>
	<textad>
		Voor goedkoop goud, klik hier!
	</textad>
\end{lstlisting}

Men moet de webpagina downloaden, dus download men de reclame vanzelf ook mee. In plaats van de reclame te blokkeren kan men hier de reclame verbergen, zodat die niet zichtbaar is voor de gebruiker. Dit is wat element hidding voor bedoeld is. Element hidding regels zijn gelijkaarding opgebouwd als request blocking regels en beginnen steeds met \texttt{\#\#}. Daarnaast worden de regels opgebouwd met behulp van \glspl{css selectors}. In bovenstaand geval kan men bijvoorbeeld als regel \texttt{\#\#div.textad} gebruiken om de eerste reclame blok te verwijderen. Voor het tweede reclameblok en specifiek voor één domein \texttt{example.com\#\#div\#sponsorad}

\subsection{Element Hiding helper}
\label{sec Element Hiding helper} 
Sommige ad blockers zoals Adblock Plus en µBlock Origin beschikken over een handige tool die toelaat om manuel elementen te verbergen met behulp van een selector. Dit laat eenvoudig toe om met de muis elementen te kiezen die men liever verborgen wil zien. De ad blocker zal dan automatisch een regel genereren die bij het geselecteerde element hoort.

\chapter{Gebruik van adblockers}
\label{ch:Gebruik van adblockers}
Eind 2015 werd wereldwijd de kaap van 200 miljoen maandelijks actieve ad block gebruikers overschreden \citep{PageFair2015}. Over volledig het jaar 2015 waren er bijna 3.2 miljard\footnote{\url{ http://www.internetlivestats.com/internet-users/\#trend}} actieve internet gebruikers. Dit betekend dat meer dan 6.25\% van de internet gemeenschap een ad blocker gebruikt. En verwacht wordt dat dit percentage de komende jaren enkel maar zal stijgen.

\section{Wie gebruikt ze}
\label{sec:Wie gebruikt ze}
De gemiddelde ad block gebruiker is een man tussen de achttien en de negenentwintig jaar en woont in een westers land \citep{PageFair2014}.
\subsection{Demografische spreiding}
\label{sec Demografische spreiding}
testttest \ref{fig: Demographic_age_sex}

\begin{figure}
\centering
\includegraphics[width=7cm]{img/demographicsMV}
\caption{Aantal ad block gebruikers volgens leeftijd en geslacht}
\label{fig: Demographic_age_sex}
\end{figure}
\begin{figure}
\centering
\includegraphics[width=7cm]{img/demographicsAge}
\caption{Aantal ad block gebruikers volgens leeftijd}
\label{fig: Demographic_age}
\end{figure}
% \includegraphics[width=2.5cm]{img/demographicsMV}\\[.5cm]



\section{Redenen voor gebruik adblocker}
\label{sec:Redenen voor gebruik adblocker}
\chapter{Huidige impact van adblockers}
\label{ch:Huidige impact van adblockers}
\chapter{Hoe wapenen bedrijven zich tegen adblockers}
\label{ch:Hoe wapenen bedrijven zich tegen adblockers}
\chapter{Gevolgen op lange termijn}
\label{ch:Gevolgen op lange termijn}
\chapter{Hoe zien de makers van adblockers de toekomst}
\label{ch:Hoe zien de makers van adblockers de toekomst}
%% TODO: de structuur en titel van deze hoofdstukken hangen af van je
% eigen onderzoek. Elke fase in je onderzoek kan een eigen hoofdstuk krijgen. Kies telkens een gepaste titel. ``Corpus'' is *GEEN* gepaste titel

\chapter{Conclusie}
\label{ch:conclusie}

% TODO: Trek een duidelijke conclusie, in de vorm van een antwoord op de
% onderzoeksvra(a)g(en). Reflecteer kritisch over het resultaat. Zijn er
% zaken die nog niet duidelijk zijn? Heeft het ondezoek geleid tot nieuwe
% vragen die uitnodigen tot verder onderzoek?



\bibliographystyle{apa}
\bibliography{tin-bachproef}

%%---------- Back matter -------------------------------------------------

\listoffigures
\listoftables

\end{document}
